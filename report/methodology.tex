Describe your method in detail and with great clarity, distinguishing it from
other works (if it is indeed a novel idea). It is very important to clearly
motivate your method.

Describe the results of your method here in this chapter.

\section{Datasets}

Crous~\cite{crous2019polyphonic} has some data on Xeno-canto.

Since all the models described in Section~\ref{sec:classification} are trained
in a supervised learning fashion, this requires a dataset of labelled training
examples. The website xeno-canto.org (XC) houses the largest and most
comprehensive publicly available collection of birdsong samples in the world. It
has made an enormous impact in the field birdsong recognition since its
inception in 2005 and has been source of the datasets for the annual BirdCLEF
challenge since 2014~\cite{vellinga2015xeno}. All samples available on XC are
labelled with the bird species and contain rich metadata, such as the time and
location of the recording. Importantly, all samples have a crowd-sourced rating
from A --- E which signifies how clear the recording sample is, where A denotes
samples with the highest quality and clarity. This is especially important
should we wish to experiment with different signal-to-noise ratios (SNR) as the
noise can be manually added to a clean recording at precise SNR levels. 

Due to its scale and reputation in the birdsong classification community, all
samples used in this thesis have been downloaded from the XC repository. All
samples downloaded have an A rating and have been labelled as a `song' rather
than a `call'. The samples are stored in a directory labelled according to the
first three letter of species' binomial name. For example, samples from the
common blackbird (\textit{Turdus merula}) are stored in a directory called
`TURMER'. This directory name acts as a label for training and test samples.

In this work we are mostly concerned with testing relative improvements in model
performance after tuning various components, such as novel feature
representations. To this end, we select the most simple form of classification
experiment: binary classification. The two bird classes selected for all
experiments herein are the common blackbird and the common nightingale
(\textit{Luscinia megarhynchos}) due to the two birds' vocalizations being
acoustically similar and hence more difficult to distinguish resulting in more
variable accuracy results, and the author's personal preference.

\section{Segmentation}

Recordings downloaded from XC were also chosen based on their length. Recordings
of duration 40 seconds to 120 seconds were preferred since they were likely to
be long enough to include enough variety of vocalizations from one individual
bird, but not too long so as to take up too much space on disk. In this thesis
it was preferred to have fewer samples from more individual as opposed to more
samples from fewer individuals for each species. The intention of this was to
introduce more variation in the training samples for each class, especially when
considering that many birds of the same species but residing in different
locations have demonstrated subtle variations in vocalizations, known as
dialects~\cite{baker1985biology}. As a general rule of thumb, we tried to have
roughly 50 training samples per individual.

This leads to the question of how to generate samples. In the literature there
seem to be two main ways of segmenting recordings into samples to be used for
training. The first is segmenting a recording into overlapping segments of a
certain length, typically in the range of 4 --- 11 seconds
(\cite{yan2021birdsong},~\cite{crous2019polyphonic}). This has the advantage
that all samples will be of fixed length and that the segmentation algorithm is
very simple. However, it will likely mean that some samples will contain only
background noise which, if used for training, will introduce noise into the
system. These noise samples therefore may need to be removed, either
manually~\cite{yan2021birdsong} or using an
algorithm~\cite{narasimhan2017simultaneous}. The other method involves
segmenting the recording into syllables
(\cite{fagerlund2007bird},~\cite{ramashini2022robust}). This has the advantage
that all training samples are likely to contain little or no noise. However the
samples will be of variable length, so steps will be needed to be taken in order
to compare the samples, such as padding or more advanced techniques like dynamic
time warping~\cite{somervuo2006parametric}. Syllable segmentation also has the
enticing prospect of being able to identify a bird species from a very short
sample. This could be useful in situations where a recording is mostly corrupted
by background noise but has a few small segments of clear birdsong.

\section{Hypothesis 1}

Hypothesis 1 (H1) is stated as follows:

\begin{quote}
Explore some of the hyperparameters available during the feature extraction
process. The hypothesis is that using hyperparameter values more suited to
birdsong classification problems will yield a higher classification accuracy.
\end{quote}

H1 was applied to two feature extraction processes that appear in some form
widely in the literature related to birdsong classification: MFCC and GTCC\@.

\subsection{MFCC}

The key parameter under test in this work as mentioned in
Section~\ref{sssec:mfcc} is the \textit{BandEdges} argument. 7 experiments were
devised and evaluated in order to see if any improvements to binary bird
classification models utilising MFCC could be made with adjusted
\textit{BandEdges} arguments. The experiments are listed in
Table~\ref{table:h1_mfcc_experiments}.

\begin{table}[h!t]
\begin{center}
\begin{tabular}{c c c c}
\toprule
Number & Number of bands & Band type & Approximate frequency range (Hz) \\ [0.5ex]
\midrule
1 & 40 & Mel & 133 --- 6864 \\
2 & 40 & Mel & 50 --- $\text{fs}/2$ \\
3 & 80 & Mel & 50 --- $\text{fs}/2$ \\
4 & 40 & Linear & 50 --- $\text{fs}/2$ \\
5 & 40 & Linear & 133 --- 6864 \\
6 & 40 & A-mel & 133 --- 6864 \\
7 & 40 & Mel & 133 --- 10000 \\
\bottomrule
\end{tabular}
\caption{Description of experiments used for testing H1 with
MFCC.}\label{table:h1_mfcc_experiments}
\end{center}
\end{table}

\subsubsection{Comments on experiment setup}

In the table, `fs' refers to the frequency sampling rate for a given audio file,
usually 44.1 KHz or 48 KHz. $\text{fs}/2$ is the maximum possible value for the
\textit{BandEdges} argument. `A-mel', short for anti-mel, denotes a set of mel
filterbanks that have been inverted, such that the center frequencies are closer
together at higher frequencies and further apart at lower frequencies. `Linear'
refers to a set of filterbanks that have been spaced linearly on the frequency
scale. The different band types can be visualized in figure

<figure of the different band types>

The experiment design was motivated as follows:

\begin{itemize}

  \item [Exp 1:] These are the defaults provided with the \texttt{mfcc}
    function and acts as a control experiment.

  \item [Exp 2:] A wider frequency range may be able to make use of the
    higher frequency harmonics produced by most birdsong.

  \item [Exp 3:] A wider frequency range means the bands have a larger bandwidth
    and may lose some distinguishing power. Increasing the number of bands may
    help to alleviate this.

  \item [Exp 4:] Since the human auditory system hasn't evolved specifically to be
    able to distinguish birdsong, it's entirely possible that the mel scale
    isn't the optimum scale to use. A linearly spaced set of filterbanks may
    provide a good generalised attempt at distinguishing birdsong.

  \item [Exp 5:] Similar to Experiment 4 but focused on a frequency range that
    more tightly fits around the dominant frequencies produced by bird
    vocalizations.

  \item [Exp 6:] The anti-mel scale has been used with interesting results in
    human speech recognition through telephones~\cite{lei2009mel}. It offers
    more distinguishing power at higher frequencies, and since birdsong is
    typically produced at higher frequencies than human speech, the anti-mel
    scale may be able to better distinguish birdsong.

  \item [Exp 7:] Similar to Experiment 2 but still focused on the main
    frequencies emitted by birds.

\end{itemize}

\subsubsection{Results}

A chart of both the AUC and the classification accuracy for MFCC can be seen in
figure~\ref{fig:hyp1_mfcc}.

\begin{figure}[ht]
  \centering
  \includegraphics[width=\textwidth]{figures/hyp1_mfcc.png}
  \caption{The AUC and classification accuracy scores for linear and RBF models
  for different experiments for MFCC.}\label{fig:hyp1_mfcc}
\end{figure}

As shown in the chart, both evaluation metrics are highest in the control
experiment. Experiment 5 returns comparable metrics to the control experiment,
with slight improvements made to the classification accuracy for the RBF model.
The RBF model shows a clear reduction in performance for wider frequency ranges
across both evaluation metrics.

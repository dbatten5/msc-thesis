The main contribution of this study is the investigation of important
hyperparameters used to extract classic features like MFCC and GTCC, and the
application of MRCG to birdsong classification. We have shown that the values
selected for hyperparameters can have a significant effect on the overall
classification accuracy and that the hyperparameter values should be carefully
considered for the problem at hand. In terms of birdsong, we have shown that
designing filter banks that span a range of $\backsim$1 kHz --- 10 kHz to be
used for MFCC and GTCC extraction yields higher model performance. During this
investigation we have shown that MFCC usually outperforms GTCC when the sample
recordings are clean.

While the MRCG results achieved here were less than desirable, it's unclear
whether this is from the model design or the suitability of MRCG with birdsong
as some concessions had to be made to allow for MRCG's significantly higher
dimensionality compared to classic features such as MFCC\@. This could be
verified by testing MRCG with some more complex CNN or CRNN model architectures
on more powerful machines.

We have also investigated some more flexible and modern architectures such as CNN
and RNN and have confirmed that they can easily outperform more classical models
like SVM\@. In doing so we have also proposed a novel method of generating
training and testing samples by combining syllables, the elemental blocks of
birdsong, into sequences. We have also laid the theoretical groundwork for an
alternative form of sequence generation that can be used with CRNN in future work.

Birds play a crucial role in ecosystems around the world. They form an important
link in the food chain, pollinate plants, and even plant
trees~\cite{broughton2021long}. The quantity and diversity of birds observed in
an area can therefore be seen as a key indicator of the strength of its
ecosystem.

Aside from being important ecological agents, birds also colour our lives with
their sights, sounds and behaviours. The community of birdwatchers, known
colloquially in the U.K.~as `twitchers', has grown considerably over the past
few years. The Royal Society for the Protection of Birds (RSPB) reported a 70\%
increase in their website views over the first lockdown, with more than 50\% of
those views on pages looking at bird identification. The Bird Bird Garden Watch,
an annual event which encourages people to note bird sightings in their own
residences, brought in 1 million participants in January 2021, more than double
the previous year's tally. As such there is a growing commercial demand for
easily accessible bird identification tools.

Birds are typically shy and protective creatures and tend to reside out of
harm's way in shrubs, trees and nests, and therefore are usually heard but not
seen. The consequence of this is that birds are typically identified through
their vocalizations, known as birdsong, rather than their visual sighting.
Birdsong used for identification is usually recorded on microphones that may be
running for a long time, so recordings may be corrupted from a wide range of
sources, such as ambient background noise, changes in birdsong amplitude, long
periods of silence, and vocalizations from other birds. There is therefore a
need for robust birdsong identification tools that can handle these corruptions
and that require minimal human intervention in order to classify unknown
birdsong.

As with most machine learning classification problems, the work boils down to
two key components. The first is feature extraction, that is, given an input
signal usually in the form of an amplitude varying over time, how

There has been numerous research 

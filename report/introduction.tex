Birds play a crucial role in ecosystems around the world. They form an important
link in the food chain, pollinate plants, and even plant
trees~\cite{broughton2021long}. The quantity and diversity of birds observed in
an area can therefore be seen as a key indicator of the strength of its
ecosystem.

Aside from being important ecological agents, birds also colour our lives with
their sights, sounds and behaviours. The community of birdwatchers, known
colloquially in the U.K.~as `birders' or `twitchers', has grown considerably
over the past few years. The Royal Society for the Protection of Birds (RSPB)
reported a 70\% increase in their website views over the first lockdown, with
more than 50\% of those views on pages looking at bird identification. The Bird
Bird Garden Watch, an annual event in the U.K. that encourages people to note
bird sightings in their residences, brought in 1 million participants in January
2021, more than double the previous year's tally. As such there is a growing
commercial demand for accurate and easily accessible bird identification tools.

Birds are typically shy and protective creatures and tend to reside out of
harm's way, obscured in shrubs, trees and nests. The majority of their
communication is done through distinctive vocalisations, known as birdsong, that
are usually unique to an individual species. Due to this, birds are typically
identified through their birdsong rather than their visual sighting. Birdsong
used for identification is usually recorded on microphones that may be running
for a long time and/or located in a place that isn't necessarily optimum to
capture the vocalisation. The recordings may be corrupted from a wide range of
sources, such as ambient background noise, large changes in birdsong amplitude,
long periods of silence, and vocalisations from other birds or animals.
Recordings captured by microphones may be several hours in duration so it would
be impractical to have a human expert identify the birds manually. There is
therefore a need for robust birdsong identification tools that can handle these
corruptions and that require minimal human intervention in order to classify
unknown birdsong. 

Modern solutions to this problem are increasingly relying on machine learning
and as with most machine learning classification problems, the work boils down
to two key components. The first is feature extraction, that is, given an input
signal usually in the form of an amplitude varying over time, how can it be
transformed to a vector or matrix representation that captures the key
information of the signal. The second is classification, i.e.~given this
representation in the feature space, how can it be used to train a model that
can then perform classification on unseen samples of birdsong. There are further
steps that can be taken to improve this process, such as pre-processing the
signal in order to remove or reduce the influence of background
noise~\cite{potamitis2014automatic}.

There has been a lot of research into audio classification problems both in
birdsong and in other types of audio, such as speaker identification from human
speech~\cite{KUDO19991103}. However, there remain novel methods that have been
tested and have been shown to have good results in wider audio classification
problems that have yet to be tried out in birdsong identification problems.
These methods include both feature extraction and classification. The
broad aim of this thesis is to attempt to evaluate these methods. In more
detail, the aims of this thesis can be summarised as follows:

\begin{enumerate}

  \item Explore some of the hyperparameters available during the feature
    extraction process. The hypothesis is that using hyperparameter values more
    suited to birdsong classification problems will yield a higher classification
    accuracy.

  \item Explore the performance of deep learning architectures such as Recurrent
    Neural Networks (RNN) and Convolutional Neural Networks (CNN) and compare
    the results with simpler statistical models. The hypothesis is that more
    complex and flexible architectures such as these will have superior
    performance when compared to simpler statistical models, such as SVMs.

  \item Explore the birdsong classification performance of feature
    representations shown to have promising results for non-birdsong related
    audio classification problems. The hypothesis is that feature
    representations shown to have good results in audio classification problems
    will have good results for birdsong identification.

\end{enumerate}
